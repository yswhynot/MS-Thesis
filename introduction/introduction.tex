\documentclass[../main.tex]{subfiles}

\begin{document}
Multi-robot systems have been widely studied for their ability to perform collective behaviors to accomplish complicated tasks, for example, environment exploration \cite{burgard2005coordinated}, cooperative sampling \cite{luo2018adaptive}, pursuit and evasion \cite{hollinger2009efficient}, search and rescue \cite{kantor2003distributed}. Planning and controlling multi-robot systems have been challenging due to the high dimentionality of state space, which would increase computation exponentially with the increasing number of robots within the system. In this thesis, I will present two ways of handling the high dimentionality problem: abstraction and specialization. 

For the abstraction, we propose a hybrid hierarchical structure for planning under uncertainty. The application scenario we consider is the pursuit-evasion problem in indoor environments, which have received significant attention owing to the complex structure that makes the problem more difficult. We also consider multiple pursuers as utilizing collaborative peers for pursuit increases the probability of catching the evader. However, (i) the limited sensor view of pursuers caused by stationary obstacles and (ii) the coordination with peer pursuers make the problem complex and difficult to model using simple representations.

%\textit{(Related work: Existing approaches and their limitations)}
One of the early work~\cite{hollinger2009efficient} formulated the problem as a multi-agent graph search problem. Although its efficient performance and scalability for locating a target object, the graph search approach is not sufficiently rich to model the uncertainty from the target that could disappear from the sensing range of the pursuers after they observe the target. Partially Observable Markov Decision Processes (POMDPs) can incorporate the uncertainty caused by the target that could disappear from the pursuers' view since POMDPs allow the state variables that cannot be directly observed. For example, the pose of the target may not be directly observed as it can move freely while it is not in the line of sight of the pursuers. Although POMDPs is capable of modeling such targets, the state space of POMDPs grows exponentially with the number of robots which makes the approach intractable.

Thus, we introduce a scalable decision-making framework for the multi-robot pursuit-evasion problem, which is \textit{Hybrid Hierarchical Partially Observable Markov Decision Processes (HHPOMDPs)}. Our framework creates an abstraction of the environment and reduces the state space for improved scalability. We define three planning states in our framework: \textit{Base MDP states}, \textit{Transition states}, and \textit{Abstract POMDP states}. The base MDPs are defined for the cases where the evader is visible to the pursuers. The abstract POMDPs provide decision-making processes for the pursuers if the evader states are not directly observable as the evader can be out of the sensing areas of the pursuers. The transition states provide an interface to switch between the base MDPs and abstract POMDPs. For the abstract POMDP state, we model the environment with convex hulls to reduce the size of state space. The base MDP states are the states from the grid world directly. The algorithm terminates when one of the pursuers is at the same location in the grid world as the evader.

Compared with standard POMDP approaches, our HHPOMDP method creates a hierarchical structure that utilizes environmental characteristics to reduce the complexity of the state space. 
Our approach of modeling the map with convex hulls and borders reduces the size of state space significantly compared with simple grid world models, while still maintain sufficient information for planning as well as incorporating the uncertainty of the target.

In the rest of this paper, we will define the problem formally in Sec.~\ref{sec:prob_def}, present the detailed algorithm in Sec.~\ref{sec:HHPOMDP}, and finally present experimental results and comparisons in Sec.~\ref{sec:result}. 

With a growing number of robots available for performing tasks, specialization is helpful for controlling the robots to achieve their goals. In continuous space and real-world applications, each robot within the system usually has a limited range of communication \cite{moreau2005stability} and is only able to exchange information with its neighbors in the connectivity graph. Maintaining connectivity of the whole system is essential \cite{hsieh2008maintaining} since the collaborative behaviors rely highly on the connection between robots \cite{zavlanos2008distributed} and it takes extensive amount of work to restore the connection once it is lost \cite{mi2011hero}. In this network robotics system, distributed algorithms, where each robot reasons and controls using only the local neighbor information, is also important for the performance and scalability \cite{zavlanos2008distributed}.

Most work focuses on maintaining connectivity based on existing control laws \cite{zavlanos2008distributed, hollinger2010multi}, or study the connectivity from a given behavior or state \cite{sabattini2013distributed, olfati2007consensus}. These are based on the assumption that every robot within the system is assigned a set of behaviors to perform in sequence \cite{nagavalli2017automated} or in parallel. Due to connectivity constraints or environmental limitations, some of the robots might not be able to perform the desired behavior. For example, when some robots are assigned the behavior of flocking north and the other robots are assigned to flock south, the system could end up with some robots staying in the middle to maintain connectivity of the whole system. However, it may be more efficient to add robots whose primary task is to maintain the connectivity of the whole system. In other words, these robots would not have any task behavior assigned to them. We define this kind of robot as \textit{connection robots}. Accordingly, we define the robots with assigned behaviors to be \textit{task robots}. With support from \textit{connection robots}, the task robots with assigned behavior may have more flexibility in achieving their goals without being overly constrained by connectivity maintenance.

In this part, our main goal is to design and analyze the controller for connection robots to maintain a flexible connectivity graph with \textit{provably faster convergence rate} of the whole system, so as to support robots with assigned behaviors to perform as desired. The challenges for connection robots are 1) keeping up with the task robots; 2) avoiding blocking the task robots, and 3) providing fast convergence rate. We will present \textit{weighted rendezvous}, \textit{weighted flocking}, and \textit{weighted behavior combination} that combines weighted rendezvous and weighted flocking to deal with the above-mentioned challenges. We will show, both theoretically and experimentally, that our method is able to provide flexible connectivity for the task robots to perform their assigned behavior by distributedly correcting the topology of the connectivity graph.

In this thesis, I will present the results of both methods on various scenarios and maps to show that both algorithms are efficient and scalable.
\end{document}