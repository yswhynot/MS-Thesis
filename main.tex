%for a more compact document, add the option openany to avoid
%starting all chapters on odd numbered pages
\documentclass[12pt]{cmuthesis}

% This is a template for a CMU thesis.  It is 18 pages without any content :-)
% The source for this is pulled from a variety of sources and people.
% Here's a partial list of people who may or may have not contributed:
%
%        bnoble   = Brian Noble
%        caruana  = Rich Caruana
%        colohan  = Chris Colohan
%        comar    = Cyrus Omar
%        jab      = Justin Boyan
%        josullvn = Joseph O'Sullivan
%        jrs      = Jonathan Shewchuk
%        kosak    = Corey Kosak
%        mjz      = Matt Zekauskas (mattz@cs)
%        pdinda   = Peter Dinda
%        pfr      = Patrick Riley
%        dkoes = David Koes (me)

% My main contribution is putting everything into a single class files and small
% template since I prefer this to some complicated sprawling directory tree with
% makefiles.

% some useful packages
\usepackage{amssymb}
\usepackage{amsmath}
\usepackage{amsthm, algorithm, algorithmicx}
\usepackage{authblk}
\usepackage[noend]{algpseudocode}
\usepackage{array}
\usepackage[english]{babel}
\usepackage{caption}
\usepackage{colortbl}
\usepackage{dsfont}
% \usepackage{enumitem}
\usepackage{fullpage}
\usepackage{float}
\let\labelindent\relax
\usepackage{gensymb} 
\usepackage{graphicx}
\usepackage[utf8]{inputenc}
\usepackage{multicol}
\usepackage{multirow}
\usepackage[numbers,sort]{natbib}
\usepackage[backref,pageanchor=true,plainpages=false, pdfpagelabels, bookmarks,bookmarksnumbered,
%pdfborder=0 0 0,  %removes outlines around hyper links in online display
]{hyperref}
\usepackage{pgfkeys}
\usepackage{subcaption}
\usepackage{subfiles}
\usepackage{tabu}
\usepackage{times}
\usepackage{tikz}
\usepackage{xparse}
\usepackage{ctable}

\usetikzlibrary{positioning}
\graphicspath{ {img/} }
% Approximately 1" margins, more space on binding side
%\usepackage[letterpaper,twoside,vscale=.8,hscale=.75,nomarginpar]{geometry}
%for general printing (not binding)
\usepackage[letterpaper,twoside,vscale=.8,hscale=.75,nomarginpar,hmarginratio=1:1]{geometry}
\algnewcommand\algorithmicinput{\textbf{Input:}}
\algnewcommand\algorithmicoutput{\textbf{Output:}}
\algnewcommand\Input{\item[\algorithmicinput]}%
\algnewcommand\Output{\item[\algorithmicoutput]}%

\theoremstyle{plain}
\newtheorem{theorem}{Theorem}
\newtheorem{preposition}[theorem]{Preposition}
\newtheorem{lemma}[theorem]{Lemma}
\newtheorem{definition}[theorem]{Definition}

% Provides a draft mark at the top of the document.
\draftstamp{\today}{DRAFT}

\begin{document} 
\frontmatter

%initialize page style, so contents come out right (see bot) -mjz
\pagestyle{empty}

\title{ %% {\it \huge Thesis Proposal}\\
{\bf Improving Scalability in Multi-Robot Systems with Abstraction and Specialization}}
\author{Sha Yi}
\date{May 2019}
\Year{2019}
\trnumber{}

\committee{
Katia Sycara, Chair \\
Stephen Smith \\
George Kantor \\
Wenhao Luo
}

\support{}
\disclaimer{}

% copyright notice generated automatically from Year and author.
% permission added if \permission{} given.

\keywords{Multi-agent, planning, control, POMDP, connectivity}

\maketitle

\begin{dedication}
For my parents
\end{dedication}

\pagestyle{plain} % for toc, was empty

%% Obviously, it's probably a good idea to break the various sections of your thesis
%% into different files and input them into this file...

\begin{abstract}
Planning and controlling multi-robot systems is challenging during the dimensionality. When the number of the robots increases in the system, the complexity of computation grows exponentially. In this thesis, we examine the scalability problem in planning and control of multi-robot system. We propose two ideas to improve scalability of multi-robot systems: abstraction and specialization.

For planning with abstraction, we propose a Hybrid Hierarchical Partially Observable Markov Decision Processes structure for improved scalability and efficiency in uncertain environment. We focus our application on the problem of pursuit-evasion with multiple pursuers and one evader in indoor environment. Since the evader might not be visible to the pursuers, incorporating uncertainty in planning would further increase computation complexity. Our structure consists of (i) the base MDPs for the cases where the evader is visible to the pursuers, (ii) the abstract POMDPs for the evader states that are not directly observable, and (iii) the transition states bridging between the base MDPs and abstract POMDPs.
This hybrid approach significantly reduces the number of states expanded in the policy tree to solve the problem by abstracting environment structures. Experimental results show that our method expands only 5\% of nodes generated from a standard POMDP solution and improve scalability.

Specialization is important for systems with large number of robots, where each robot has an individual controller and collaboratively perform tasks. Connectivity maintenance is essential for these collaborative behaviors since robots need to communicate with neighbor robots to share states and information.
It is inefficient for a robot to both attend to task behaviors and also maintain connectivity.
Therefore, we introduce the idea of \textit{connection robots}. These robots do not have task behaviors assigned to them, but their goal is to maintain connectivity of the robot team by correcting the topology of the connectivity graph so as to provide flexibility for the task robots to perform task behaviors. We propose a distributed approach of blending weighted rendezvous and weighted flocking to form the controller for connection robots. We prove that this scheme is able to guarantee a faster convergence rate. We then present our result of a system of up to 30 robots in various cluttered environments and show that our approach is robust and scalable. 
\end{abstract}

\begin{acknowledgments}
I would like to express my sincere gratitude to my advisor Prof. Katia Sycara for the continuous support of my Master study and research for her patience, motivation, and immense knowledge. Her guidance helped me in all the time of research and preparing for this thesis. Although being always busy, she is always available for discussion when I am stuck or have new ideas. I would not be able to accomplish my achievement without her help.

Besides my advisor, I would like to thank the rest of my thesis committee for their insightful comments and encouragement, but also for the questions that widened my research from various perspectives.

I thank my fellow labmates in for the stimulating discussions, for the sleepless nights we were working together before deadlines, and for all the fun we have had in the last two years. Special thank to Wenhao Luo, who has answered all my weird questions and provided valuable insights into my research problems.

I would also like to thank my boyfriend Gengshan Yang for getting me awesome food when I get depressed, and being a wonderful companion inside and outside academic studies.

Last but not the least, I would like to thank my  parents for supporting me spiritually throughout my pursuit of academic career and my my life in general.
\end{acknowledgments}

\tableofcontents
\listoffigures

\mainmatter

%% Double space document for easy review:
%\renewcommand{\baselinestretch}{1.66}\normalsize

% The other requirements Catherine has:
%
%  - avoid large margins.  She wants the thesis to use fewer pages, 
%    especially if it requires colour printing.
%
%  - The thesis should be formatted for double-sided printing.  This
%    means that all chapters, acknowledgements, table of contents, etc.
%    should start on odd numbered (right facing) pages.
%
%  - You need to use the department standard tech report title page.  I
%    have tried to ensure that the title page here conforms to this
%    standard.
%
%  - Use a nice serif font, such as Times Roman.  Sans serif looks bad.
%
% Other than that, just make it look good...


\chapter{Introduction}
\subfile{introduction/introduction.tex}

\chapter{Related Work} \label{chap:related_work}
\subfile{related_work/related_work.tex}

\chapter{Plan with Abstraction} \label{chap:abstraction}
\subfile{abstraction/abstraction.tex}

\chapter{Control with Specialization} \label{chap:special}
\subfile{special/special.tex}

\chapter{Conclusion and Future Work}
\subfile{conclusion/conclusion.tex}
%\appendix
%\include{appendix}

\backmatter

%\renewcommand{\baselinestretch}{1.0}\normalsize

% By default \bibsection is \chapter*, but we really want this to show
% up in the table of contents and pdf bookmarks.
\renewcommand{\bibsection}{\chapter{\bibname}}
%\renewcommand{\bibpreamble}{This text goes between the ``Bibliography''
%  header and the actual list of references}
\bibliographystyle{plainnat}
\bibliography{ref.bib} %your bib file

\end{document}
